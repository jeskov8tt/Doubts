\documentclass{article}
\usepackage{graphicx}
\usepackage{amsmath}
\usepackage{amssymb}
\usepackage[italicdiff]{physics}
\usepackage{enumerate}
\usepackage{microtype}
\DisableLigatures{encoding= *, family=*}
\usepackage{titlesec}
\usepackage{xfrac}
\setcounter{secnumdepth}{4}
\usepackage{xcolor}
\usepackage{hyperref}
\hypersetup{
    colorlinks=true,
    linkcolor=[RGB]{59 108 209},
    urlcolor=[RGB]{59 108 209}
}
\urlstyle{same}

\titleformat{\paragraph}
{\normalfont\normalsize\bfseries}{\theparagraph}{1em}{}
\titlespacing*{\paragraph}
{0pt}{3.25ex plus 1ex minus .2ex}{1.5ex plus .2ex}

\title{}
\author{}
\date{}

\begin{document}
\maketitle

\section{Doubts}
\begin{enumerate}
      \item Arihant Amit M. Agrawal, Continuity and Differentiability, Session2, Exercise, Q3 "How is $f$ continuous at $x=0$"
      \item Arihant Amit M. Agrawal, Continuity and Differentiability, Session3, Exercise, Q2, (b) "How is $g(x)$ a function?"
      \item Arihant Amit M. Agrawal, Continuity and Differentiability, Session5, Exercise, Q1, "How is the given function discontinuous at $x=0$"
      \item Arihant Amit M. Agrawal, Continuity and Differentiability, JEE Type Examples, Example 2
      \item Amit M. Agrawal, Continuity and Differentiability, JEE Type Examples, Example 11
      \item Amit M. Agrawal, Continuity and Differentiability, JEE Type Examples, Example 12
      \item Amit M. Agrawal, Continuity and Differentiability, JEE Type Examples, Example 22
      \item If $f'(x)$ is given then how to identify differentiability of $f$?
      \item If $f(x)$ is a polynomial and has $n$ real roots then $f'(x)$ has $n-1$ real roots?
      \item Amit M. Agrawal, Continuity and Differentiability, JEE Type Examples, Example 31
      \item Amit M. Agrawal, Differentiatiion, Session3, Shortcut for Differentiation of Implicit Functions
      \item Disprove,

            If $$f(x)=\displaystyle\sum_{n=1}^{x} x=\underbrace{x+x+x+\ldots +x}_{x \hspace{1mm} \text{times}}$$
            then $$f'(x)=\displaystyle\sum_{n=1}^{x} 1=\underbrace{1+1+1+\ldots+1}_{x \hspace{1mm} \text{times}}=x$$
            $\therefore$ $$\dv{x}(f(x))=\dv{x}(x^2)=x$$
      \item Neighbourhood in limits
      \item \href{https://math.stackexchange.com/questions/4892287/is-this-result-on-continuity-of-composite-functions-true}{Math StackExchange Question on Continuity of Composite Functions}
      \item \href{https://math.stackexchange.com/questions/689575/proof-that-every-polynomial-of-odd-degree-has-one-real-root}{Math StackExachange Proof on Roots of Odd Degree Polynomials } \\ Doubt in answer by  \href{https://math.stackexchange.com/users/91982/shuchang}{Shuchang} in Method of IVT.
      \item Let $f(x)=x^2$ and $g(x)=\displaystyle\sum_{n=1}^{x}x=\underbrace{x+x+x+\ldots +x}_{x \hspace{1mm} \text{times}}$
            \\ are both $f$ and $g$ same?
      \item Consider the funciton $$f(x)=\underbrace{x+x+x+\ldots+x}_{x \hspace{1mm} \text{times}}$$ \\
            What is domain of $f$?
      \item Consider, $$F(x)=max\left\{f_{1}(x),f_{2}(x),f_{3}(x)\right\} \hspace{1mm} \forall \hspace{1mm} x \in \mathbb{R}$$

            If $f_{1}(x) > f_{2}(x) \hspace{1mm} \forall \hspace{1mm} x \in \mathbb{R}$

            Then, Prove or Disprove

            $$F(x)=max\left\{f_{1}(x),f_{3}(x)\right\}$$
      \item Prove or Disprove,
            If $f(x)$ is a real continous function for all $x \in \mathbb{R}$ and $f$ is symmetric about two different lines perpendicular to axis of $x$ (say $x=a$ and $x=b$, $a>b$), i.e. $$f(a-x)=f(a+x) $$ and $$f(b-x)=f(b+x)$$ then $f$ is periodic with period $2(a-b)$
      \item Prove or Disprove,
            $$\displaystyle\int_{0}^{a} f(x) dx=\displaystyle\int_{0}^{a} f(a-x) dx$$
      \item Prove or Disprove, 
      $$\lim\limits_{x \to 0^+}{\dfrac{\cos^{-1} \left(1-x^2\right)}{x}}=\sqrt{2}$$
      \item The range of $$f(x)=8^{\cos x}+8^{\sin x}$$
\end{enumerate}
\end{document}